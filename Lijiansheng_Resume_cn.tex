% (c) 2002 Matthew Boedicker <mboedick@mboedick.org> (original author) http://mboedick.org

% (c) 2003-2007 David J. Grant <davidgrant-at-gmail.com> http://www.davidgrant.ca

% (c) 2008 Nathaniel Johnston <nathaniel@nathanieljohnston.com> http://www.nathanieljohnston.com

% (c) 2011 Scott Clark <sc932@cornell.edu> http://cam.cornell.edu/~sc932
%

%This work is licensed under the Creative Commons Attribution-Noncommercial-Share Alike 2.5 License. To view a copy of this license, visit http://creativecommons.org/licenses/by-nc-sa/2.5/ or send a letter to Creative Commons, 543 Howard Street, 5th Floor, San Francisco, California, 94105, USA.

\documentclass[letterpaper,11pt]{article}
%\documentclass[12pt,a4paper,UTF8,adobefonts]{ctexart}

\newlength{\outerbordwidth}

\pagestyle{empty}

\raggedbottom

\raggedright

\usepackage[svgnames]{xcolor}

\usepackage{framed}

\usepackage{tocloft}

%-----------------------------------------------------------
% chinese font setting
\usepackage[utf8]{inputenc}
\usepackage{fontspec,xunicode}
%% \usepackage[slantfont, boldfont, CJKtextspaces, CJKmathspaces]{xeCJK} % 允许斜体和粗体
\usepackage[slantfont,boldfont]{xeCJK} % 允许斜体和粗体
% \usepackage[SlantFont,BoldFont,CJKtextspaces,CJKmathspaces]{xeCJK} % 允许斜体和粗体
\setCJKmainfont{WenQuanYi Micro Hei} % 设置缺省中文字体
\setCJKmonofont{WenQuanYi Micro Hei Mono} % 设置等宽字体
\setmainfont{Tahoma} % 英文衬线字体
\setsansfont{Tahoma} % 英文无衬线字体
\setmonofont{Monaco} % 英文等宽字体
\usepackage{indentfirst} % 首段缩进
\defaultfontfeatures{Mapping=tex-text} %如果没有它,会有一些 tex 特殊字符无法正常使用,比如连字符。
% 中文断行
\XeTeXlinebreaklocale "zh"
\XeTeXlinebreakskip = 0pt plus 1pt minus 0.1pt

%Edit these values as you see fit

\setlength{\outerbordwidth}{3pt}  % Width of border outside of title bars

\definecolor{shadecolor}{gray}{0.75}  % Outer background color of title bars (0 = black, 1 = white)

\definecolor{shadecolorB}{gray}{0.93}  % Inner background color of title bars

%-----------------------------------------------------------

%Margin setup

\setlength{\evensidemargin}{-0.25in}

\setlength{\headheight}{-0.25in}

\setlength{\headsep}{0in}

\setlength{\oddsidemargin}{-0.25in}

\setlength{\paperheight}{11in}

\setlength{\paperwidth}{8.5in}

\setlength{\tabcolsep}{0in}

\setlength{\textheight}{9.75in}

\setlength{\textwidth}{7in}

\setlength{\topmargin}{-0.3in}

\setlength{\topskip}{0in}

\setlength{\voffset}{0.1in}

%-----------------------------------------------------------

%Custom commands

\newcommand{\resitem}[1]{\item #1 \vspace{-2pt}}

\newcommand{\resheading}[1]{\vspace{8pt}

  \parbox{\textwidth}{\setlength{\FrameSep}{\outerbordwidth}

    \begin{shaded}

\setlength{\fboxsep}{0pt}\framebox[\textwidth][l]{\setlength{\fboxsep}{4pt}\fcolorbox{shadecolorB}{shadecolorB}{\textbf{\sffamily{\mbox{~}\makebox[6.762in][l]{\large #1} \vphantom{p\^{E}}}}}}

    \end{shaded}

  }\vspace{-5pt}

}

\newcommand{\ressubheading}[4]{

\begin{tabular*}{6.5in}{l@{\cftdotfill{\cftsecdotsep}\extracolsep{\fill}}r}

		\textbf{#1} & #2 \\

		\textit{#3} & \textit{#4} \\

\end{tabular*}\vspace{-6pt}}

%-----------------------------------------------------------

\begin{document}

\begin{tabular*}{7in}{l@{\extracolsep{\fill}}r}

\textbf{\Large 李建盛} & \textbf{birthday:1982/4} \\

lijiangsheng1@gmail.com \\
slideshare:johnwoolee & mobile:+86 18910219825\\
twitter:lijiangsheng1 \\

\end{tabular*}

%%%%%%%%%%%%%%%%%%%%%%%%%%%%%%

\resheading{Education}

%%%%%%%%%%%%%%%%%%%%%%%%%%%%%%

\begin{itemize}

\item

	\ressubheading{School Of Distance Learning Peking University}{Beijing}{B.S. Computer Science}{2008 - 2012}

	\begin{itemize}

		\resitem{}


	\end{itemize}

\item

	\ressubheading{ZhongZhou University}{ZhengZhou,henan}{Associate Technology of Network}{2002 - 2005}

	\begin{itemize}

		\resitem{Strong emphasis on scientific computing, network tech. and software development}

	\end{itemize}

\end{itemize}

%%%%%%%%%%%%%%%%%%%%%%%%%%%%%%

\resheading{Skills}

%%%%%%%%%%%%%%%%%%%%%%%%%%%%%%

\begin{itemize}

\item Ten years Linux user and 5 years developer.

\item {\bf Development:} Python/Perl/Ruby, C, Java, bash, xml

\item {\bf Computer Science:} Operating Systems, Cluster(HA,LB,Distributed), TCP/IP protocols,

\item {\bf IT architecture:} LAMP J2EE SOA 

\item {\bf Cloud computing:} OpenStack,Eucalyptus,OpenNebula,CloudStack,Aeolus (install,deploy and troubeshooting)

\item {\bf Virtualization:} KVM,qemu,Ovirt,libvirt

\item {\bf Storage:} SAN/NAS, LVM,Distributed/Object file system

\item {\bf OpenSouce Project:} xCat,zenoss,PostgreSQL,JBoss(wildfly)

\item Discovering and implementing new ideas. Hacking.

\item I have worked in many places in a myriad of fields. I can readily learn and adapt to a new discipline, area or environment and start pushing real results quickly.

\end{itemize}

%%%%%%%%%%%%%%%%%%%%%%%%%%%%%%

\resheading{Research and Work Experience}

%%%%%%%%%%%%%%%%%%%%%%%%%%%%%%

\begin{itemize}

\item

	\ressubheading{cloud-times Inc}{Beijing}{Development Manager}{Jun 2011 - current}

	\begin{itemize}

		\resitem{Build strong high performance developer team.}
                \resitem{Based Linux Virtualization tech. KVM, and SPICE project.}
        \resitem{Active in Open Source community, Like \texttt{Ovirt,libivrt,glusterFS} etc.Community-based development model, reference fedora/RHEL model, and encourage team members to participate in community development.}
                \resitem{Practice of Scurm/XP/Agile.}
                \resitem{used git+trac+jenkins for project management.}
                \resitem{extend project: Customized Linux operating systems for thin client, based OpenSuse build service platforms.}

	\end{itemize}

\item

	\ressubheading{iSoft Inc}{Beijing}{Senior Software Engineer}{Mar 2010 - Jun 2011}

	\begin{itemize}

		\resitem{The IAAS platform demand analysis, open source selection, team building, process definition, refactoring etc.}

                \resitem{To the magazine 《Linux Pilot》 published about Eucalyptus and Convirt. based research/test.}

                \resitem{used program language: \texttt{Java},\texttt{C},\texttt{python},\texttt{bash}}
               
               \resitem{Version control: SVN. library: gcc/glib/libxml/libvirt/ etc based open source software: CloudStack,libvirt,OpenVSwitch,iptables etc.}

	\end{itemize}

\item

	\ressubheading{RedFlag co.ltd., }{Beijing}{Software Engineer}{Jun 2008 - Mar 2010}

	\begin{itemize}

		\resitem{Responsible for Red Flag Linux server operating system installation,high-availability cluster software development.}

    \resitem{used program language: \texttt{C},\texttt{python/bash}}
    \resitem{Version control: SVN/mercurial}
    \resitem{library: gcc/glib/libxml etc}
    \resitem{based open source software: anaconda/heartbeat/peacemaker}
    \resitem{and a some research/test system: Hudoop, Virtualization, and port XtreemOS from slackware to Redflag.}

	\end{itemize}

\item 

	\ressubheading{RedFlag co.ltd.,}{Beijing}{Technical Support/Maintenance Engineer}{Nov 2006 - Jun 2008}

	\begin{itemize}

		\resitem{Tech. Support for RedFlag enterprise linux, Deployment, troubleshooting, training, solutions for customer.}

    \resitem{project experience:}
    \resitem{0、Deployment: China Postal Savings Bank pre-server, China Postal UnionPay, National Weather Service HA cluster, China Post philatelic system HA Cluster}
    \resitem{1、Troubleshooting:China Postal Center, The Yunnan Agricultural Bankcheck system HA,}
    \resitem{2、solutions: China Postal Savings Bank Share file system ,China Postal Disaster system SCSI device bonding.}

	\end{itemize}

\item

	\ressubheading{Henan Newspapering Network Center(dahe.cn)}{ZhengZhou, henan}{Infrastructure Management}{Jun 2005 - Oct 2006}

	\begin{itemize}

		\resitem{outine maintenance, include X86 arch hardware, OS(Redhat/RedFlag linux, Windows server 2003),DNS(bind), Web server(apahce,IIS,Tomcat),Application Server(weblogic,websphere)RDBMS(MySQL,SQL server,DB2)and website application (CMS, Search, video, etc.)}

	\end{itemize}

\item


\end{itemize}

%%%%%%%%%%%%%%%%%%%%%%%%%%%%%%
\resheading{Participation in Workshops and Conferences}
%%%%%%%%%%%%%%%%%%%%%%%%%%%%%%
\begin{itemize}
\item
	\ressubheading{BPUG 2010}{douban, beijing}{Python in Virtualization}{November 18, 2010}
\item
	\ressubheading{Cloud Valley World 2012}{beijing, China}{How to build geeks team}{Decmber 13, 2012}
\end{itemize}

%%%%%%%%%%%%%%%%%%%%%%%%%%%%%%

\resheading{Selected Open Source Projects and Publications {\mdseries(github.com/lijiangsheng1)}}

%%%%%%%%%%%%%%%%%%%%%%%%%%%%%%

\begin{itemize}

\item \ressubheading{OVDG: Open Virtualization Develop Guide}{xml, xslt,bash}{the book for junior developer}{2012 - Current}

  \begin{itemize}

    \resitem{based docbook for book}

    \resitem{in Chinese publican,No one book about virtulaization write for junior developer.}

	\end{itemize}

\item \ressubheading{glusterfs-appliance}{Python, Bash, Perl}{A project forked RedHat Storage}{2013 - Current}

  \begin{itemize}

		\resitem{centos + glusterfs packages(node + management) -- rpm repo -- kickstart file -- imagefacory -- appliance(ec2,openstack,ovirt,vshpere etc.)}

	\end{itemize}


\end{itemize}

\resheading{Personal}

\begin{itemize}

\item {\bf Hobbies:} Reading,Movies and drinking good beer or scotch with new and old friends.

\item {\bf My Ideal Position:} Working with a fun team solving interesting problems. I enjoy every part of development, from deep backend optimization to client-facing applications and interaction. 

\end{itemize}

\end{document}
